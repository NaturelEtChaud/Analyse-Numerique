\section{Exercices}
\begin{exer}[Scilab ?]$ $\\
\begin{tabularx}{\textwidth}{m{0.2\textwidth}m{0.7\textwidth}}
\begin{center}
	\includegraphics[scale=1]{00-Calculer-avec-une-machine/images/scilab.png}
\end{center}
&
T�l�charger le document \og Scilab pour les vrais d�butants\fg{} soit en recopiant l'adresse suivante :\newline
 \url{www.scilab.org/fr/content/download/849/7897/file/Scilab_debutant.pdf},\newline
soit en demandant poliment � votre moteur de recherche favori.\newline
Tester les commandes et les codes propos�s.
\\
\end{tabularx}
\end{exer}

\begin{exer}(\%eps)
Taper dans la console de Scilab les commandes suivantes (sauf les commentaires) :
\begin{lstlisting}
-->format(17) //pour obtenir une pr�cision de 17 d�cimales
-->%eps
-->%eps/2
-->1+%eps/2
-->1+%eps/2==1
-->-1-%eps/2
-->-1-%eps/2==-1
-->1-%eps/2
-->1-%eps/2==1
-->%eps*1e16
\end{lstlisting}
\end{exer}


\begin{exer}[Un premier programme]
H�ron d'Alexandrie\footnote{H�ron d'Alexandrie est un ing�nieur, un m�canicien et un math�maticien grec du I\up{er} si�cle apr. J.-C.} a imaginer une m�thode permettant de d�terminer une approximation de la racine carr�e d'un nombre entier positif. On pense que les Babyloniens utilisaient d�j� cette m�thode\footnote{Pour plus d'information sur le sujet, voir la tablette YBC 7289.} d�s le II\up{e} mill�naire av. J.-C.\\
Avec nos notations modernes, si l'on souhaite obtenir une approximation de $\sqrt{a}$ avec $a>0$, il \og suffit\fg{} de programmer la suite
$$\begin{cases} L_0&=... \\
								L_{n+1}&=\dfrac{1}{2}\left(L_n + \dfrac{a}{L_n}\right) \end{cases}$$
en prenant pour $L_0$ le nombre de votre choix (� condition qu'il soit positif et non nul). Nous choisirons par d�faut $L_0=1$.
\begin{enumerate}
	\item Programmer une fonction ayant pour arguments $a$ et $n$.\\
				Tester le pour obtenir une approximation de $\sqrt{2}$ puis de $\sqrt{16}$.
	\item Am�liorer votre programme pour afficher le nuage de points correspondants aux $n$ termes de la suite calcul�s.
	\item Am�liorer votre programme en ajoutant en arguments la pr�cision voulue dans le r�sultat.\\
				Pour cela vous comparerez � chaque �tape le r�sultat obtenu par la m�thode de H�ron avec le calcul $\sqrt{a}$ effectuer directement par Scilab.
\end{enumerate}
\end{exer}

\begin{center}	\includegraphics[scale=0.7]{00-Calculer-avec-une-machine/images/humour2.png}\end{center}


