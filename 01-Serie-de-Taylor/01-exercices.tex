\section{Exercices}
\setcounter{exer}{0}

\begin{exer}[Polyn�me de Taylor]
\begin{enumerate}
	\item Trouver le polyn�me de Taylor � l'ordre 3 en 0 de la fonction exponentielle.
	\item V�rifier que les d�riv�es $k$\up{�me} du polyn�me obtenu et de la fonction exponentielle co�ncident en 0, pour $0\leq k \leq 3$.
\end{enumerate}
\end{exer}

\begin{exer}[Polyn�me de Taylor suite ...]
En utilisant votre polyn�me de Taylor � l'ordre 3 au voisinage de 0 de la fonction exponentielle, calculer une approximation de $e^{0,1}$ et majorer l'erreur commise.\\
De m�me calculer une approximation de $e^{1}$ avec le m�me polyn�me de Taylor. Commenter le r�sultat.
\end{exer}

\medskip

\begin{exer}[... et fin]
�crire la formule de Taylor-Lagrange � l'ordre 4 pour les fonctions suivantes dans un voisinage de 0.
\begin{enumerate}
\begin{tabularx}{\textwidth}{XXX}
\item $f(x)=\dfrac{1}{1-x}$ & \item $g(x)=\ln(1+x)$ & \item $h(x)=\sin(2x)$ \\
\end{tabularx}
\end{enumerate}
\end{exer}


%%%%%%%%%%%%%%%%%%%%%%%%%%%%%%%%%%%%%%%%%%%%%%%%%%%%%
%%%%%%%%%%%%%%%%%%%%%%%%%%%%%%%%%%%%%%%%%%%%%%%%%%%%%%%%%%%%%%%
\begin{exer}[Scilab]
Trouver � partir de quel rang $n$ le terme $\dfrac{0,1^n}{n!}$ est inf�rieur � \textbf{\%eps}.
\end{exer}


\begin{exer}[Polyn�me de Taylor avec Scilab]
Tracer dans un m�me graphique la fonction exponentielle et son polyn�me de Taylor � l'ordre 3 au voisinage de 0 sur l'intervalle $[-1;1]$ puis sur l'intervalle $[-5;5]$. 
\end{exer}

\begin{exer}[Fonction homographique avec Scilab]
Une fonction homographique est une fonction de la forme $x\mapsto \dfrac{ax+b}{cx+d}$ avec $a,b,c$ et $d$ des r�els.\\
\begin{enumerate}
	\item Notons $g$ la fonction homographique correspondant au cas $a=0, b=d=1$ et $c=-1$.\\
				Quel est l'ensemble de d�finition de la fonction $g$ ?
	\item Calculer $g'(x), g''(x), g^{(3)}(x)$ et plus g�n�ralement $g^{(n)}(x)$.
	\item �crire le plus simplement possible $T_{5,0}(x)$, le polyn�me de Taylor � l'ordre 5 au voisinage de 0 de la fonction $g$.
	\item Sur un m�me graphique, donner l'allure des courbes repr�sentatives de $g$ et de $T_{5,0}$ avec Scilab.
	\item Soit $h(x)=\dfrac{5-3x}{1-x}$.\\
				Montrer qu'il existe deux r�els $\alpha$ et $\beta$ tels que $h(x)=\alpha + \dfrac{\beta}{1-x}$.\\
				En d�duire $U_{5,0}(x)$, le polyn�me de Taylor � l'ordre 5 au voisinage de 0 de la fonction $h$ pour $x$ appartenant � un bon intervalle.
\end{enumerate}
\end{exer}


